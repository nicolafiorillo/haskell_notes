\documentclass[a4paper, 12pt]{book}

\usepackage[utf8]{inputenc}
\usepackage[italian]{babel}
\usepackage[T1]{fontenc}
\usepackage{titleps}
\usepackage{graphicx}
\usepackage{caption}
\usepackage[protrusion=true,expansion=true]{microtype}
\usepackage[colorlinks]{hyperref}
\usepackage{import}
\usepackage{bookmark}

\graphicspath{{images/}}

\import{}{commands.tex}
\import{}{page_style.tex}

\begin{document}

    \import{}{cover.tex}
    \tableofcontents

    \chapter{Introduzione}

    Definiamo \textbf{immagine} una funzione bidimensionale \(f(x, y)\) dove \(x\) e \(y\) sono coordinate spaziali e \(f\) è l'ampiezza (\emph{intensità}) di ogni coppia di coordinate.\\
    Definiamo \textbf{immagine digitale} quando i valori di \(f\) sono finiti.\\

\end{document}
